%%%%%%%%%%%%%%%%%%%%%%%%%%%%%%%%%%%%%%%%%%%%%%%%%%%%%%%%%%%%%%%%%%
%%%%%%%% ICML 2017 EXAMPLE LATEX SUBMISSION FILE %%%%%%%%%%%%%%%%%
%%%%%%%%%%%%%%%%%%%%%%%%%%%%%%%%%%%%%%%%%%%%%%%%%%%%%%%%%%%%%%%%%%

% Use the following line _only_ if you're still using LaTeX 2.09.
%\documentstyle[cs541,epsf,natbib]{article}
% If you rely on Latex2e packages, like most moden people use this:
\documentclass{article}

% use Times
\usepackage{times}
% For figures
\usepackage{graphicx} % more modern
%\usepackage{epsfig} % less modern
\usepackage{subfigure} 

% For citations
\usepackage{natbib}

% For algorithms
\usepackage{algorithm}
\usepackage{algorithmic}

% As of 2011, we use the hyperref package to produce hyperlinks in the
% resulting PDF.  If this breaks your system, please commend out the
% following usepackage line and replace \usepackage{cs541} with
% \usepackage[nohyperref]{cs541} above.
\usepackage{hyperref}

% Packages hyperref and algorithmic misbehave sometimes.  We can fix
% this with the following command.
\newcommand{\theHalgorithm}{\arabic{algorithm}}

\usepackage[accepted]{cs541}


% The \icmltitle you define below is probably too long as a header.
% Therefore, a short form for the running title is supplied here:
\icmltitlerunning{Title of My Paper}

\begin{document} 

\twocolumn[
\icmltitle{Title of My Paper}

% It is OKAY to include author information, even for blind
% submissions: the style file will automatically remove it for you
% unless you've provided the [accepted] option to the icml2017
% package.

% list of affiliations. the first argument should be a (short)
% identifier you will use later to specify author affiliations
% Academic affiliations should list Department, University, City, Region, Country
% Industry affiliations should list Company, City, Region, Country

% you can specify symbols, otherwise they are numbered in order
% ideally, you should not use this facility. affiliations will be numbered
% in order of appearance and this is the preferred way.
\icmlsetsymbol{equal}{*}

\begin{icmlauthorlist}
\icmlauthor{Aeiau Zzzz}{wpi}
\icmlauthor{Bauiu C.~Yyyy}{wpi}
\end{icmlauthorlist}

\icmlaffiliation{wpi}{Worcester Polytechnic Institute}

\icmlcorrespondingauthor{Cieua Vvvvv}{c.vvvvv@wpi.edu}

% You may provide any keywords that you 
% find helpful for describing your paper; these are used to populate 
% the "keywords" metadata in the PDF but will not be shown in the document
\icmlkeywords{boring formatting information, machine learning, ICML}

\vskip 0.3in
]

% this must go after the closing bracket ] following \twocolumn[ ...

% This command actually creates the footnote in the first column
% listing the affiliations and the copyright notice.
% The command takes one argument, which is text to display at the start of the footnote.
% The \icmlEqualContribution command is standard text for equal contribution.
% Remove it (just {}) if you do not need this facility.

\printAffiliationsAndNotice{}  % leave blank if no need to mention equal contribution
%\printAffiliationsAndNotice{\icmlEqualContribution} % otherwise use the standard text.
%\footnotetext{hi}

\begin{abstract} 
Describe {\bf briefly} the computational problem you tackled, the approach you explored, the analyses you performed, and the results you obtained.
Point out advantages of your approach compared to prior approaches (if any).
\end{abstract} 

\section{Introduction}
{\bf Problem before solution}:
Begin by describing a (computational) {\bf problem} and why the problem is {\bf important}. If previous research has
explored plausible solutions to solve the problem, then point out how these approaches may be lacking. Then briefly {\bf motivate}
why a new approach (the approach you explore in this paper) is worth exploring. Don't go into too much detail yet how it works -- that will
be left for Section \ref{sec:proposed_method}).

\subsection{Research contributions}
Describe clearly and concisely what your paper {\bf uniquely} contributes to the research community, i.e., that has never been done before.

\section{Related Work}
Describe previous approaches (in somewhat greater detail than the introduction) to tackling the problem. Pick any citation style you prefer and use it
consistently (here's a random citation \cite{mitchell80}). Explain how previous work may be lacking in certain ways, and use your critical review
of prior work to further motivate your research.

\section{Proposed Method}
\label{sec:proposed_method}
Describe the computational approach -- based obviously on neural networks -- that you took to tackle the problem.

\section{Experiment}
Describe the computational experiments you conducted to validate the approach you took.

\section{Results}
Describe the results of the experiments, including their immediate implications (e.g., ``the result suggests that technique A performs better
than technique B''). If your work is purely theoretical, then this section (as well as the previous section) might be replaced with mathematical proofs.

\section{Discussion}
Describe limitations of your work as well as further-ranging implications (e.g., ``our work suggests that doing XYZ in general may be a useful approach to...'').

\section{Conclusions and Future Work}
Briefly summarize the key findings your paper and point out interesting directions for future inquiry.

\bibliography{paper}
\bibliographystyle{cs541}

\end{document} 


% This document was modified from the file originally made available by
% Pat Langley and Andrea Danyluk for ICML-2K. This version was
% created by Lise Getoor and Tobias Scheffer, it was slightly modified  
% from the 2010 version by Thorsten Joachims & Johannes Fuernkranz, 
% slightly modified from the 2009 version by Kiri Wagstaff and 
% Sam Roweis's 2008 version, which is slightly modified from 
% Prasad Tadepalli's 2007 version which is a lightly 
% changed version of the previous year's version by Andrew Moore, 
% which was in turn edited from those of Kristian Kersting and 
% Codrina Lauth. Alex Smola contributed to the algorithmic style files.  
